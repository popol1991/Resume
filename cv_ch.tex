%%%%%%%%%%%%%%%%%%%%%%%%%%%%%%%%%%%%%%%%%
% Medium Length Professional CV
% LaTeX Template
%
% This template has been downloaded from:
% http://www.LaTeXTemplates.com
%
% Original author:
% Trey Hunner (http://www.treyhunner.com/)
%
% Important note:
% This template requires the resume.cls file to be in the same directory as the
% .tex file. The resume.cls file provides the resume style used for structuring the
% document.
%
%%%%%%%%%%%%%%%%%%%%%%%%%%%%%%%%%%%%%%%%%

%-------------------------------------------------------------------------------------
%	PACKAGES AND OTHER DOCUMENT CONFIGURATIONS
%-------------------------------------------------------------------------------------

\documentclass{resume} % Use the custom resume.cls style

\usepackage[BoldFont,SlantFont,CJKchecksingle]{xeCJK}
\setCJKmainfont[BoldFont=SimHei]{SimSun}
\setCJKmonofont{SimSun}% 设置缺省中文字体

\usepackage[left=0.75in,top=0.6in,right=0.75in,bottom=0.6in]{geometry} % Document margins

\name{高英恺} % Your name
\address{15011179836 popol1991@gmail.com} % Your phone number and email 

\begin{document}
{\bf 意向: }兼职软件开发工程师。
{\bf 可实习时间: }03/2014 - 07/2014

%-------------------------------------------------------------------------------------
%	EDUCATION SECTION
%-------------------------------------------------------------------------------------

\begin{rSection}{教育背景}

{\bf 北京航空航天大学} \hfill {\em Sep.2010 - Present} \\ 
计算机学院 \\
本科三年级 平均分: 90.15 排名: 3/217\\
{\bf 新加坡国立大学} \hfill {\em Aug.2012 - Dec.2012} \\
计算机学院 \\
导师: 蔡达成教授

\end{rSection}

%-------------------------------------------------------------------------------------
%	WORK EXPERIENCE SECTION
%-------------------------------------------------------------------------------------

\begin{rSection}{实践经验}

\begin{rSubsection}{微软亚洲研究院}{Apr.2013 - Jan.2014}{研究实习生}{知识挖掘组}
\item 提出了一个弱监督的聚类框架用以自动收集关于实体属性的搜索查询表达方式,并将各查询表达方式与知识库中的实体属性关联起来。
\end{rSubsection}

\begin{rSubsection}{NLP\&CC 2012 微博情感分析评测}{Jun.2012 - Nov.2012}{开发\&研究}{北京航空航天大学智能信息研究所}
\item 开发了用以文本情感分析的程序框架,提供分词、依存分析、命名实体识别等预处理服务。实现了作为对照实验的baseline情感分析程序。
\item 与53个队伍比较,在观点句识别任务中排名第一,观点句极性分类排名第四。
\end{rSubsection}

%------------------------------------------------

\begin{rSubsection}{NExT搜索中心}{Aug.2012 - Dec.2012}{实习开发}{新加坡国立大学计算机学院}
\item 参与了实时用户生成内容爬虫的开发,该爬虫由Java语言和MongoDB在一个集群上开发,可以实时爬取Twitter,新浪微博,Flickr,Instagram,点评网等站点的内容。
\end{rSubsection}

%------------------------------------------------

\begin{rSubsection}{社交网络环境中的情感分析}{Dec.2011 - Present}{发起人\&开发}{北京航空航天大学智能信息研究所}
\item 成立了一个五人小组,开发了人人网应用“情感过山车”,该应用利用SVM分类器分析用户的情感变化并与用户进行互动。
\item 该项目由中国大学生创新创业项目资助。
\end{rSubsection}

%------------------------------------------------

\begin{rSubsection}{微策略MicroStrategy}{Jun.2012 - July.2012}{实习软件工程师}{北京}
\item 使用Java以及Spring Framework,参与开发用于大规模电子商务的Web Service后端。
\end{rSubsection}

\end{rSection}

\pagebreak[4]

%----------------------------------------------------------------------------------
%	TECHNICAL STRENGTHS SECTION
%----------------------------------------------------------------------------------

\begin{rSection}{技术能力}

{\bf 擅长:} \\
\begin{tabular}{ @{} >{\hspace{6ex}\bfseries}l @{\hspace{6ex}} l }
基础知识 & 算法, 数据结构, 机器学习, 设计模式 \\
编程语言 & Java (20k+ 行代码), C \\
工具/库 & Spring Framework, Maven, Git, SVN, Sublime Text
\end{tabular}

{\bf 熟悉:} \\
\begin{tabular}{ @{} >{\hspace{6ex}\bfseries}l @{\hspace{6ex}} l }
基础知识 & 自然语言处理(情感分析), 数据挖掘, 网络协议\\
编程语言 & Bash, Javascript, Python, Prolog, Haskell, HTML, CSS \\
数据库 & MySQL, MongoDB \\
工具/库 & jQuery, LaTeX
\end{tabular}

\end{rSection}

%----------------------------------------------------------------------------------
%	Honors/Awards
%----------------------------------------------------------------------------------
\begin{rSection}{荣誉奖励}
微软小学者 (Aug.2013) \\
\hspace*{8pt} - 全国36人 \\
中国计算机学会优秀大学生 (Oct.2013) \\
\hspace*{8pt} - 全国100人 \\
北京航空航天大学三星奖学金 (Oct.2012) \\
国家奖学金 (Oct.2011) \\
北京航空航天大学学习优秀一等奖学金, 文艺优胜一等奖学金 (Oct.2011) \\
第21 届北京航空航天大学“冯如杯”学生科技作品竞赛创新奖 (Mar.2011) \\
第三届世界大学生摄影展 最佳摄影奖 (Apr.2011) \\
全国高中信息学奥林匹克联赛 一等奖 (Dec.2008)
\end{rSection}



%-------------------------------------------------------------------------------------
%	Qualities&Other Skills
%-------------------------------------------------------------------------------------

\begin{rSection}{特点与兴趣}
\begin{tabular}{ @{} >{\bfseries}l @{\hspace{6ex}} l }
特点 & 有效的解决问题能力, 自发式学习, 高效的人际交流能力\\
英语 & 流利使用正式英语与口语\\
& 托福分数: 110 \\
活动 & 北航计算机学院学生会副主席 \\
& Coursera, TED翻译志愿者 \\
& CCF\&ACM 学生会员 \\
& 北航程序设计第二课堂学生教师 \\
& 美国职业摄影师协会会员 \\
& 业余钢琴十级
\end{tabular}
\end{rSection}

%-------------------------------------------------------------------------------------

\end{document}
