%%%%%%%%%%%%%%%%%%%%%%%%%%%%%%%%%%%%%%%%%
% Medium Length Professional CV
% LaTeX Template
%
% This template has been downloaded from:
% http://www.LaTeXTemplates.com
%
% Original author:
% Trey Hunner (http://www.treyhunner.com/)
%
% Important note:
% This template requires the resume.cls file to be in the same directory as the
% .tex file. The resume.cls file provides the resume style used for structuring the
% document.
%
%%%%%%%%%%%%%%%%%%%%%%%%%%%%%%%%%%%%%%%%%

%-------------------------------------------------------------------------------------
%	PACKAGES AND OTHER DOCUMENT CONFIGURATIONS
%-------------------------------------------------------------------------------------

\documentclass{resume} % Use the custom resume.cls style

\usepackage[BoldFont,SlantFont,CJKchecksingle]{xeCJK}
\setCJKmainfont[BoldFont=SimHei]{SimSun}
\setCJKmonofont{SimSun}% 设置缺省中文字体

\usepackage[left=0.75in,top=0.6in,right=0.75in,bottom=0.6in]{geometry} % Document margins

\name{高英恺} % Your name
\address{15011179836 popol1991@gmail.com} % Your phone number and email 

\begin{document}
{\bf 意向: }内容编辑类实习职位 \hspace{2em}
{\bf 可实习时间: }03/2014 - 07/2014

%-------------------------------------------------------------------------------------
%	EDUCATION SECTION
%-------------------------------------------------------------------------------------

\begin{rSection}{教育背景}

{\bf 北京航空航天大学} \hfill {\em Sep.2010 - Present} \\ 
计算机学院 \\
本科四年级 平均分: 90.15 排名: 3/217\\
{\bf 新加坡国立大学} \hfill {\em Aug.2012 - Dec.2012} \\
计算机学院 

\end{rSection}

%-------------------------------------------------------------------------------------
%	WORK EXPERIENCE SECTION
%-------------------------------------------------------------------------------------

\begin{rSection}{实践经验}

\begin{rSubsection}{微软亚洲研究院}{Apr.2013 - Jan.2014}{研究实习生}{知识挖掘组}
在完成技术工作的同时,与运营组和公共关系组保持紧密关系,熟悉微软夏令营、实习生新年晚会等活动的策划、组织、宣传与实施工作。
\end{rSubsection}

%------------------------------------------------

\begin{rSubsection}{北京航空航天大学计算机学院学生会}{Sep.2010 - July.2012}{副主席}{}
\item 主管负责计算机学院的学生科技实践活动,计算机协会。组织学生参加中国大学生创新创业训练计划、大
学生科研训练计划等活动。北航10、11级军训视频摄影、制作人。
\end{rSubsection}

%------------------------------------------------

\begin{rSubsection}{微策略MicroStrategy}{Jun.2012 - July.2012}{实习软件工程师}{北京}
\item 使用Java以及Spring Framework,参与开发用于大规模电子商务的Web Service后端。
\end{rSubsection}

\end{rSection}

%----------------------------------------------------------------------------------
%	TECHNICAL STRENGTHS SECTION
%----------------------------------------------------------------------------------

\begin{rSection}{技术能力}

\begin{tabular}{ @{} >{\hspace{6ex}\bfseries}l @{\hspace{6ex}} l }
工具 & Photoshop, Premiere \\
技术 & HTML, CSS, JavaScript
\end{tabular}

\end{rSection}

%----------------------------------------------------------------------------------
%	Honors/Awards
%----------------------------------------------------------------------------------
\begin{rSection}{荣誉奖励}
第三届世界大学生摄影展 最佳摄影奖 (Apr.2011) \\
微软小学者 (Aug.2013) \\
\hspace*{8pt} - 全国36人 \\
中国计算机学会优秀大学生 (Oct.2013) \\
\hspace*{8pt} - 全国100人 \\
北京航空航天大学三星奖学金 (Oct.2012) \\
国家奖学金 (Oct.2011) 
\end{rSection}

%-------------------------------------------------------------------------------------
%	Qualities&Other Skills
%-------------------------------------------------------------------------------------

\begin{rSection}{特点与兴趣}
\begin{tabular}{ @{} >{\bfseries}l @{\hspace{6ex}} l }
特点 & 有效的解决问题能力, 快速学习, 高效的人际交流能力\\
英语 & 流利使用正式英语与口语\\
& 托福分数: 110 \\
活动 & 美国职业摄影师协会会员 \\
& TED\&Coursera志愿翻译者 \\
& CCF\&ACM 学生会员 \\
& 北航程序设计第二课堂学生教师 \\
& 业余钢琴十级
\end{tabular}
\end{rSection}

%-------------------------------------------------------------------------------------

\end{document}
